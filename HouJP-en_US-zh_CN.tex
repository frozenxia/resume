% !TEX TS-program = xelatex
% !TEX encoding = UTF-8 Unicode
% !Mode:: "TeX:UTF-8"

\documentclass{resume}
\usepackage{zh_CN-Adobefonts_external} % Simplified Chinese Support using external fonts (./fonts/zh_CN-Adobe/)
%\usepackage{zh_CN-Adobefonts_internal} % Simplified Chinese Support using system fonts
\usepackage{linespacing_fix} % disable extra space before next section
\usepackage{cite}
\usepackage[colorlinks,linkcolor=blue]{hyperref}

\begin{document}
\pagenumbering{gobble} % suppress displaying page number


\name{Jianpeng Hou}
\centerline{Machine Learning; Software Development}
\vspace{2ex}
% {E-mail}{mobilephone}{homepage}
% be careful of _ in emaill address
% {E-mail}{mobilephone}
% keep the last empty braces!
\contactInfo{houjp1992@gmail.com}{(+86) 152-0000-0000}{ \url{https://github.com/houjp/}}

 
\section{\faGraduationCap\ Education}
\datedsubsection{\textbf{Institute of Computer Technology, Chinese Academy of Sciences}}{Sep. 2014 -- Jul. 2017}
\textit{Master student} in Computer Software \& Theory, Rank: top 5\%
\datedsubsection{\textbf{University of Science \& Technology Beijing}}{Sep. 2010 -- Jul. 2014}
\textit{Bachelor student} in Computer Science \& Technology, Rank: \nth{4}/124

\section{\faUsers\ Project Experience}
\datedsubsection{\textbf{Big Data Analysis Platform (\href{http://159.226.40.104:18080}{http://159.226.40.104:18080})}}{Oct. 2015 -- Mar. 2016}
%\role{Summer Intern}{Manager: xxx}
\begin{itemize}
  \item  Developed distributed algorithms(CART/GBDT/GBRT/RF) on Spark.
  \item Finished data mining components(Feature-Indexing/Feature-Merging/Feature-Normalization/Scoring).
%  \item Optimized xxx 5\%
\end{itemize}

\datedsubsection{\textbf{China Telecom Big Data Application Contest (\nth{1}/1112; Team Leader)}}{Dec. 2015 -- Mar. 2016}
%\role{C, Python, Django, Linux}{Individual Projects, collaborated with xxx}
The goal of this contest is to predict views of users with ten sites succeeding, according to four hundred million user-behavior historical records(25.38G).
%Brief introduction: xxx
\begin{itemize}
  \item Proposed and implemented a multi-target regression algorithm on Spark. Optimized F1-score 0.8\%.
  \item Designed and developed a probability ranking model for user classification. Optimized F1-score  0.6\%.
%  \item xxx
\end{itemize}

\datedsubsection{\textbf{SIGHAN-2015 Chinese Spelling Check Task (\nth{1} Place)}}{Mar. 2015 -- May. 2015}
%\role{\LaTeX, Maintainer}{Individual Projects}
The goal of this task is to detect and correct spelling errors on Chinese essays.
\begin{itemize}
  \item Handled this task with a unified framework which consisted of candidate generating, two stage candidates re-ranking and global decision making.
  \item Finished candidate generating model and two stage candidates re-ranking model. Optimized F1-score 18\%.
\end{itemize}

\section{\faSitemap\ Intern Experience}

\datedsubsection{\textbf{DiDi Research Institute}}{Jul. 2016 -- Sep. 2016}
%\role{\LaTeX, Maintainer}{Individual Projects}
%The goal of this task was to detect and correct spelling errors on Chinese essays.
\begin{itemize}
  \item Developed taxi dispatching system which used to balance the supply and demand between urban areas.
  \item Reconstructed the log system of dispatching based on Kafka and MySQL.
\end{itemize}

\datedsubsection{\textbf{Baidu Online Network Technology(Beijing) Co.,Ltd}}{Dec. 2013 -- May. 2014}
%\role{\LaTeX, Maintainer}{Individual Projects}
%The goal of this task was to detect and correct spelling errors on Chinese essays.
\begin{itemize}
  \item Completed the multi-threaded development of LiveWDBBroom, improved the efficiency by two times.
  \item Completed the development of the Problems-Tracing Platform which was put into service.
  \item Passed the examination of Code Master(C++), certified as Good Coder.
\end{itemize}


% Reference Test
%\datedsubsection{\textbf{Paper Title\cite{zaharia2012resilient}}}{May. 2015}
%An xxx optimized for xxx\cite{verma2015large}
%\begin{itemize}
%  \item main contribution
%\end{itemize}


\section{\faTrophy\ Academic Competitions}
\datedline{\textit{\nth{1} Place}~~Awarded in China Telecom Big Data Application Contest}{Mar. 2016}
\datedline{\textit{\nth{1} Place}~~Awarded in SIGHAN-2015 Chinese Spelling Check Task}{Jun. 2015}
\datedline{\textit{\nth{1} Place}~~Awarded in China College Students Computer Games Competition}{Nov. 2013}
\datedline{\textit{\nth{1} Prize}~~Awarded in China College Students Computer Application Contest}{Nov. 2013}
\datedline{\textit{\nth{1} Prize}~~Awarded in "LanQiao Cup" Software Development Contest(Beijing Division)}{Apr. 2012}
\datedline{\textit{Silver Medal~(\nth{17} / 200)}~~Awarded in ACM-ICPC Asia Beijing Regional Contest}{Nov. 2015}

\section{\faCogs\ Skills}
\begin{itemize}[parsep=0.5ex]
  \item Skilled in C++, Scala, Shell. Familiar with data structures and algorithms and had good programming style.
  \item Experienced in development of distributed machine learning algorithms.
  \item Strong theoretic knowledge on data mining and machine learning.
\end{itemize}



\name{夏忠林 }
\centerline{求职意向: 机器学习 | 广告算法}
% {E-mail}{mobilephone}{homepage}
% be careful of _ in emaill address
\vspace{1ex}
\contactInfo{tinymindx@hotmail.com}{(+86) 186-1126-9795}{ {http://frozenxia.github.io/} }


% {E-mail}{mobilephone}
% keep the last empty braces!
%\contactInfo{xxx@yuanbin.me}{(+86) 131-221-87xxx}{}

\vspace{-1ex}
 
\section{\faGraduationCap\  教育背景}
\datedsubsection{\textbf{武汉大学}~~ \ 硕士, 计算机应用与技术}{2012 -- 2015}

\datedsubsection{\textbf{武汉大学}~~ \ 学士, 计算机科学与技术}{2008 -- 2012}
\vspace{1ex}

%\vspace{-1ex}

\section{\faUsers\ 项目经历}

\datedsubsection{\textbf{小米~互联网商业部~/~金融策略组}}{2019.02 -- 2019.10}
%\role{Golang, Linux}{个人项目,和富帅糕合作开发}
\begin{onehalfspacing}
研发针对互联网金融用户的人群定向模型,提高激活率和注册率;为金融应用研发召回策略算法,根据不同广告位的联合回测结果,通过随即初始启发式搜索方法,在额定ROI下最优化平台收益,该人群定向模型和召回策略算法已经投入使用。
% \begin{itemize}
%   \item 实现了多维序列预测算法对销量和故障数据进行预测,与LSTM以及Prophet的预测结果相比,误差和方差更小;
% \end{itemize}
\end{onehalfspacing}

% \vspace{-1.5ex}

\datedsubsection{\textbf{小米~云平台~/~采风质量预警系统}}{2017.10 -- 2019.01}
%\role{Golang, Linux}{个人项目,和富帅糕合作开发}
\begin{onehalfspacing}
以售后、客服以及销售数据为基础研发预警算法,对产品质量问题进行快速预警;利用爬虫获取的用户评论数据,建立产品满意度指数。
\begin{itemize}
  \item 实现了多维序列预测算法,该算法与LSTM模型以及Prophet相比,对产品销量和故障数据的预测结果误差和方差更小。
  \item 实现了异常检测算法,该算法比AnomalyDetection(https://github.com/twitter/AnomalyDetection)具有更好的检测效果。
  \item 实现了多种情感分类算法(基于概率图模型和基于神经网络模型),对用户评论的情感分类准确率超过0.92。
  \item 该项目获得了2018小米质量奖三等奖。
\end{itemize}
\end{onehalfspacing}




% \vspace{-1.5ex}
\datedsubsection{\textbf{小米~云平台~/~用户画像表维护 }}{2017.08 -- 2017.10}
%\role{Golang, Linux}{个人项目,和富帅糕合作开发}
\begin{onehalfspacing}
维护公司用户画像宽表,梳理上下游数据链路,提高数据质量和可用性。经过优化之后的数据可用性从90\%提高到99.8\%,数据生成时间减少65\%。
% \begin{itemize}
%   \item Faas支持多种编程语言,包括python、java和js等。
%   \item Faas使用消息队列实现组件解耦,以etcd作为分布式锁进行全局控制,利用预加载策略提高响应速度。
% \end{itemize}
\end{onehalfspacing}



\datedsubsection{\textbf{小米~云平台~/~Faas系统 (https://open.cloud.mi.com)}}{2017.03 -- 2017.08}
%\role{Golang, Linux}{个人项目,和富帅糕合作开发}
\begin{onehalfspacing}
为小米开放云设计并实现了基于Kubernetes的Serverless框架Faas,该系统支撑了小爱同学的语音skill调用。
\begin{itemize}
  \item Faas支持多种编程语言,包括python、java和js等。
  \item Faas使用消息队列实现组件解耦,以etcd作为分布式锁进行全局控制,利用预加载策略提高响应速度。
\end{itemize}
\end{onehalfspacing}

% \vspace{-1.5ex}


\datedsubsection{\textbf{小米~云平台~/~小米生态云 ({https://cloud.mi.com})} }{2015.7 -- 2017.03}
% \role{实习}{经理: 高富帅}
为小米生态云设计并实现了用户管理模块,应用自动扩容模块,RDS和Cache的Service broker模块以及计量与计费模块,并参与了生态云的资源定价设计。
\begin{itemize}
  \item 用户管理模块将用户组作为资源的唯一关联实体,与RBAC控制策略相结合,避免了用户变更带来的资源和权限管理问题。
  \item 自动扩容模块为平台上的App提供了多种扩容方案,包括基于metrics的扩容策略和基于trigger的扩容策略,提高了用户资源的使用效率。
\end{itemize}
% \vspace{-1.5ex}


% \datedsubsection{\textbf{SIGHAN-2015 Chinese Spelling Check Task}}{2015.03 -- 2015.05}
%\role{\LaTeX, Python}{个人项目}
% \begin{onehalfspacing}
% 该评测任务是对繁体中文进行拼写检查,给出正确的拼写结果。
% \begin{itemize}
%   \item 实现候选生成排序模型:根据同音、近音、形近字,为句中每一个繁体字生成候选, 并打分排序。
%   \item 实现两轮候选重排序模型:采用简单特征进行预排序,在第一次排序的基础上结合复杂特征进行第二次排序,以此提高排序效率。最终取得第一名,F值超出第二名18\%。
% \end{itemize}
% \end{onehalfspacing}

% Reference Test
%\datedsubsection{\textbf{Paper Title\cite{zaharia2012resilient}}}{May. 2015}
%An xxx optimized for xxx\cite{verma2015large}
%\begin{itemize}
%  \item main contribution
%\end{itemize}

% \vspace{-1ex}

% \section{\faSitemap\ 实习经历}
% \datedsubsection{\textbf{~智能平台部~/~算法工程师}}{2016.07 -- 2016.09}
% \vspace{-0.5ex}
% %\role{\LaTeX, Python}{个人项目}
% \begin{onehalfspacing}
% %优雅的 \LaTeX\ 简历模板, https://github.com/billryan/resume
% \begin{itemize}
%   \item 完成专车调度系统开发、测试并上线,实现对城市各区域专车供需不平衡状态的动态调整。
%   \item 基于Kafka及MySQL完成专车调度日志系统重构,解耦子模块间相互依赖,提高系统可用性。
% \end{itemize}
% \end{onehalfspacing}

% \vspace{-1.5ex}

% \datedsubsection{\textbf{百度~网页搜索部~/~研发工程师}}{2013.12 -- 2014.05}
% \vspace{-0.5ex}
% %\role{\LaTeX, Python}{个人项目}
% \begin{onehalfspacing}
% \begin{itemize}
%   \item  完成 livewdbbroom 工具多线程开发,通过测试并上线,效率提升 2 倍以上。
%   \item 完成离线时效性日志体系建设及离线时效性问题追查平台建设,并上线。
%   \item 通过 Code Master(C++) 考试,获得 Good Coder 认证。
% \end{itemize}
% \end{onehalfspacing}

% \vspace{-1ex}

% \section{\faTrophy\ 学术竞赛}
% \datedline{\textit{冠军 (\nth{1}/1112; CNY 200,000; 队长)}~中国电信大数据算法应用大赛}{2016.03}
% \datedline{\textit{冠军}~SIGHAN-2015 Chinese Spelling Check Task}{2015.06}
% \datedline{\textit{冠军}~全国大学生计算机博弈大赛}{2013.08}
% \datedline{\textit{银牌 (\nth{17}/200)}~ACM-ICPC 国际大学生程序设计竞赛亚洲区域赛北京赛区}{2015.11}
% \datedline{\textit{四强 (\nth{4}/2293)}~阿里巴巴大数据竞赛“新浪微博互动预测大赛”}{2015.12}
% \datedline{\textit{一等奖}~华北五省及港澳台大学生计算机应用大赛}{2013.11}
% \datedline{\textit{一等奖}~全国软件专业人才设计与创业大赛北京赛区}{2012.04}

% \vspace{-1ex}

\section{\faCogs\ 个人能力}
% increase linespacing [parsep=0.5ex]
\begin{itemize}[parsep=0.5ex]
  \item 熟悉C++、Scala、java、python,熟悉基本数据结构和算法,有良好的编程风格。
  \item 有丰富的基于Spark分布式计算框架的机器学习算法开发经验。
  \item 熟悉数据挖掘、机器学习、概率图模型领域基本算法。
\end{itemize}

%\section{\faInfo\ 其他}
%% increase linespacing [parsep=0.5ex]
%\begin{itemize}[parsep=0.5ex]
%  \item 技术博客: http://blog.yours.me
%  \item GitHub: https://github.com/username
%  \item 语言: 英语 - 熟练(TOEFL xxx)
%\end{itemize}

%% Reference
%\newpage
%\bibliographystyle{IEEETran}
%\bibliography{mycite}
\end{document}
